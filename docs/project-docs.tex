\section{Cel projektu}

\textbf{Temat projektu:} {\large \textit{Deeplearning -- tagger (POS, lematyzacja)}.}
\medskip

Zadaniem realizowanego przedsięwzięcia jest stworzenie programu głębokiego uczenia (ang. \textit{deep learning}) z zakresu przetwarzania języka naturalnego. Jego celem jest przypisywanie każdemu wyrazowi w tekście wejściowym odpowiadającej mu części mowy (ang. \textit{part-of-speech, POS}). Wykonanie zadania tagowania zostanie poprzedzone procesem lematyzacji tekstu.


\begin{figure}[H]
	\centering
	\begin{tikzpicture}[node distance=3cm,auto,>=latex']
		\tikzstyle{box} = [rectangle, draw, fill=blue!10, rounded corners, inner sep=.5em]
		\tikzstyle{ball} = [ellipse, draw, fill=blue!10, align=center, anchor=north, inner sep=.5em]
		\node (a) {\textit{,,Ala ma kota''}};
		\node (b) [box, right of=a, node distance=3.5cm] {lematyzator};
		\node (c) [box, right of=b, node distance=5.5cm] {POS tagger};
		\node (d) [right of=c, node distance=3.8cm] {
			\tabcolsep=1.5pt
			\renewcommand{\arraystretch}{1}
			\begin{tabular}{ccc}
				{\footnotesize RZ} & {\footnotesize CZ} & {\footnotesize RZ} \\
				\textit{,,Ala} & \textit{mieć} & \textit{kot''}
			\end{tabular}
		};
		\node (c1) [box, above of=c, node distance=2cm] {lematyzator};
		\node (c2) [ball, above of=c1, node distance=1.5cm] {zbiór uczący};
		\path[->] (a) edge node {} (b);
		\path[->] (b) edge node {\textit{,,Ala mieć kot''}} (c);
		\draw[->] (c) edge node {} (d);
		\draw[->] (c1) edge node [anchor=center,fill=white]{uczenie} (c);
		\draw[->] (c2) edge node {} (c1);
	\end{tikzpicture}
	\caption{Uproszczony schemat działania programu.}		
\end{figure}

Realizacja powinna posiadać formę aplikacji desktopowej lub skryptu. Cel projektu zostanie osiągnięty, jeśli przygotowane oprogramowanie (wyposażone w odpowiedni zbiór uczący) będzie w stanie przetwarzać tekst w języku polskim w czasie i dokładności, które zostaną sprecyzowane przez prowadzącego.

	
\section{Koszty}
\subsection*{Szacunkowy czas wykonania}
Biorąc pod uwagę zakres tematyczny projektu szacuję się, że do jego wykonania potrzebnych jest 400 godzin roboczych. Około 1/8 czasu poświęcone będzie na zebranie i analizę materiałów dotyczących projektu, ogólne poznanie możliwości i technologii. Największa część czasu zużyta zostanie na implementację projektu (stworzenie lematyzatora oraz taggera). Szacuje się, że będzie to 6/8 czasu. Pozostała część przeznaczona jest na kontakty z prowadzącym oraz testy zaimplementowanego rozwiązania.

\subsection*{Szacunkowy koszt projektu}
Mając na uwadze poziom skomplikowania zadania projektowego stawka za godzinę pracy wynosi 30 zł. Przy szacowaniu kosztów projektu nie uwzględniamy dodatkowych wydatków, które trzeba będzie ponieść w przypadku, gdy np. zajdzie potrzeba wykupienia domeny WWW. Nie są uwzględnione także koszty wdrożenia projektu w środowisku produkcyjnym. W związku z powyższym szacunkowy koszt wykonania projektu jest równy 12 000 zł netto. 

Usługi, o których mowa w poprzednich punktach opodatkowane są 23\% stawką podatku VAT (podatek od towarów i usług). Szacowana cena brutto wynosi \textbf{14 760 zł}.

\begin{table}[H]
	\centering
	\caption{Kosztorys.}
	\smallskip
	\begin{tabular}{|c|c|c|c|c|c|c|c|}
		\hline
		\textbf{Nazwa} & \textbf{Jedn.} & \textbf{Ilość} & \textbf{Cena jedn.} & \textbf{Wart. netto} & \textbf{Stawka} & \textbf{Podatek} & \textbf{Wart. brutto} \\\hline
		Robocizna & r-g & 400 & 30 & 12 000 & 23 \% & 2 760 zł & 14 760 zł \\\hline
		\multicolumn{7}{r|}{\bf Razem:} & \textbf{14 760 zł} \\\cline{8-8}
	\end{tabular}
\end{table}
	
\section{Terminy i harmonogram projektu}
Granicznym terminem realizacji projektu jest \textbf{7 czerwca 2016 r}. W tym dniu nastąpi prezentacja rezultatów projektu wraz z przekazaniem jego pełnej dokumentacji.

\begin{table}[H]
	\centering
	\caption{Harmonogram projektu.}
	\smallskip
	\begin{tabular}{p{4cm}p{8cm}}
		\textbf{Data}& \textbf{Opis} \\\hline
		23.02.2016 & Wybór tematu projektu. \\\hline
		1.03.2016 & Określenie zakresu tematycznego projektu. \\\hline
		8.03.2016 & Przekazanie specyfikacji projektu (wstępna funkcjonalność, określenie kamieni milowych). \\\hline
		9.03.2016 -- 06.06.2016 & Konsultacja wyników pracy po osiągnięciu kolejnych kamieni milowych projektu. \\\hline
		\textbf{7.06.2016} & Prezentacja rezultatów projektu i przekazanie dokumentacji projektowej. \\\hline
	\end{tabular}
\end{table}

	\begin{table}[H]
	\centering
	\caption{Kamienie milowe.}
	\smallskip
	\begin{tabular}{p{4cm}p{8cm}}
		\textbf{Data}& \textbf{Opis} \\\hline
		29.03.2016 & Instalacja środowiska \\\hline
		19.04.2016 & Stworzenie lematyzatora \\\hline
		10.05.2016 & Stworzenie taggera \\\hline
		31.05.2016 & Stworzenie instrukcji użytkowania \\\hline
		07.06.2016 & Prezentacja projektu \\\hline		
	\end{tabular}
\end{table}