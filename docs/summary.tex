\section{Podsumowanie}
Uzyskanie wysokich skuteczności klasyfikatorów wymaga dobrania odpowiednich parametrów - liczb n-literowych końcówek. Badania wykazały, że najwyższą skuteczność klasyfikacji otrzymujemy gdy liczba 2,3,4 literowych końcówek znajuje się w przedziale [35,45], a liczba 1 literowych koncówek w przedziale [15-32]. W przypadku, gdy liczba końcówek jest zbyt duża lub zbyt mała skuteczność klasyfikacji gwałtownie spada. Oprócz tego na skuteczność klasyfikacji źle wpływają duże różnice między liczbami 2,3,4 literowych końcówek. Według przeprowadzonych różnica dwoma liczbami końcówek powinna wynosić nie więcej niż 10. Najlepszym klasyfikatorem okazał się SVM z najlepszym wynikiem 61,08\% dla korpusu PWr oraz 60,13\% dla korpusu National. Dokładność klasyfikacji można poprawić przez szerszą analizę wejściowych danych (przeprowadzoną przez specjalistów ds. języka polskiego) lub lepszy dobór parametrów.  

